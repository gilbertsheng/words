\documentclass{article}

\usepackage{xeCJK}
\usepackage{longtable}
\usepackage[a4paper,vmargin=1cm,hmargin=5mm]{geometry}

\begin{document}
\begin{center}
\begin{longtable}{|l|p{9cm}|}

\hline
\multicolumn{2}{|l|}{\textbf{Introduction}}\\

\hline
out and out
&
You use out-and-out to emphasize that someone or something has all the characteristics of a particular type of person or thing.
\\

\hline
primeval
&
You use primeval to describe things that belong to a very early period in the history of the world.
\par
You use primeval to describe feelings and emotions that are basic and not the result of thought.
\\

\hline
wiles
&
Wiles are clever tricks that people use to persuade other people to do something.
\\

\hline
channel
&
If you channel money or resources into something, you arrange for them to be used for that thing, rather than for a wider range of things.
\\

\hline
tap
&

\\

\hline
lane
&
A lane is a narrow road, especially in the country.
\\

\hline
ulcer
&
An ulcer is a sore area on the outside or inside of your body which is very painful and may bleed or produce an unpleasant poisonous substance. 溃疡
\\

\hline
insomnia
&
Someone who suffers from insomnia finds it difficult to sleep.
\\

\hline
thrash
&
If someone thrashes about, or thrashes their arms or legs about, they move in a wild or violent way, often hitting against something. You can also say that someone's arms or legs thrash about.
\\

\hline
\multicolumn{2}{|l|}{\textbf{Chapter 1 The Dinasaur Brain and Lizard Logic}}\\

\hline
stake out
&
If you stake out a position that you are stating or a claim that you are making, you are defending the boundaries or limits of the position or claim.
\\

\hline
cortex
&
The cortex of the brain or of another organ is its outer layer.
\\

\hline
limbic system
&
the part of the brain bordering on the corpus callosum: concerned with basic emotion, hunger, and sex
\\

\hline
rave
&
If someone raves, they talk in an excited and uncontrolled way.
\\

\hline
sketch
&
A sketch is a drawing that is done quickly without a lot of details. Artists often use sketches as a preparation for a more detailed painting or drawing.
\\

\hline
jam
&
If something such as a part of a machine jams, or if something jams it, the part becomes fixed in position and is unable to move freely or work properly.
\\

\hline
one-upsmanship
&
If you refer to someone's behaviour as one-upmanship, you disapprove of them trying to make other people feel inferior in order to make themselves appear more important.
\\

\hline
petty
&
You can use petty to describe things such as problems, rules, or arguments which you think are unimportant or relate to unimportant things.
\\

\hline
turf
&
Someone's turf is the area which is most familiar to them or where they feel most confident.
\\

\hline
broach
&
When you broach a subject, especially a sensitive one, you mention it in order to start a discussion on it.
\\

\hline
agitate
&
If something agitates you, it worries you and makes you unable to think clearly or calmly.
\\

\hline
prod
&
If you prod someone or something, you give them a quick push with your finger or with a pointed object.
\par
If you prod someone into doing something, you remind or persuade them to do it.
\\

\hline
extension
&
An extension is a phone line that is connected to the switchboard of a company or institution, and that has its own number. The written abbreviation ext. is also used.
\\

\hline
compare notes
&
to exchange opinions
\\

\hline
single out
&
If you single someone out from a group, you choose them and give them special attention or treatment.
\\

\hline
on merit
&
If you judge something or someone on merit or on their merits, your judgment is based on what you notice when you consider them, rather than on things that you know about them from other sources.
\\

\hline
harp on
&
If you say that someone harps on a subject, or harps on about it, you mean that they keep on talking about it in a way that other people find annoying.
\\

\hline
\multicolumn{2}{|l|}{\textbf{Chapter 2 The Rules of Lizard Logic}}\\

\hline
flamboyant
&
If you say that someone or something is flamboyant, you mean that they are very noticeable, stylish, and exciting.
\\

\hline
rabble-rouser
&
A rabble-rouser is a clever speaker who can persuade a group of people to behave violently or aggressively, often for the speaker's own political advantage.
\\

\hline
resplendent
&
If you describe someone or something as resplendent, you mean that their appearance is very impressive and expensive-looking.
\\

\hline
plaid
&
Plaid is material with a check design on it. Plaid is also the design itself. 呢子花格
\\

\hline
at a second's notice
&
Notice is used in expressions such as `on short notice,' `at a moment's notice,' or `at twenty-four hours' notice,' to indicate that something can or must be done within a short period of time.
\\

\hline
gnaw
&
If people or animals gnaw something or gnaw at it, they bite it repeatedly.
\\

\hline
appease
&
If you try to appease someone, you try to stop them from being angry by giving them what they want. 姑息
\\

\hline
visceral
&
Visceral feelings are feelings that you feel very deeply and find it difficult to control or ignore, and that are not the result of thought.
\\

\hline
pinup
&
that is or can be pinned up on or otherwise fastened to a wall
\\

\hline
reprimand
&
If someone is reprimanded, they are spoken to angrily or seriously for doing something wrong, usually by a person in authority.
\\

\hline
fall back on
&
If you fall back on something, you do it or use it after other things have failed.
\\

\hline
vow
&
If you vow to do something, you make a serious promise or decision that you will do it.
\\

\hline
bristle
&
If you bristle at something, you react to it angrily, and show this in your expression or the way you move.
\\

\hline
jab
&
If you jab one thing into another, you push it there with a quick, sudden movement and with a lot of force.
\\

\hline
ethologist
&
an expert in the study of the behaviour of animals in their normal environment
\\

\hline
\multicolumn{2}{|l|}{\textbf{Chapter 3 Get It Now!}}\\

\hline
triceratops
&
三角龙
\\

\hline
lumber
&
If someone or something lumbers from one place to another, they move there very slowly and clumsily.
\par
If you are lumbered with someone or something, you have to deal with them or take care of them even though you do not want to and this annoys you.
\\

\hline
clump
&
If someone clumps somewhere, they walk there with heavy, clumsy steps.
\\

\hline
broccoli
&
Broccoli is a vegetable with green stalks and green or purple tops. 青花菜
\\

\hline
strong(est) suit
&
a person's greatest talent, most conspicuous character trait, etc.
\\

\hline
spark plug
&
A spark plug is a device in the engine of a motor vehicle, which produces electric sparks to make the petrol burn.
\\

\hline
grind
&
If you refer to routine tasks or activities as the grind, you mean they are boring and take up a lot of time and effort.
\\

\hline
adrenaline
&
肾上腺素
\\

\hline
junkies
&
A junkie is a drug addict.
\par
You can use junkie to refer to someone who is very interested in a particular activity, especially when they spend a lot of time on it.
\\

\hline
\multicolumn{2}{|l|}{\textbf{Chapter 4 The Triple F Response: Fight, Flight or Fright}}\\

\hline
abrasive
&
Someone who has an abrasive manner is unkind and rude.
\\

\hline
animosity
&
Animosity is a strong feeling of dislike and anger.
\\

\hline
innate
&
An innate quality or ability is one that a person is born with.
\\

\hline
procrastination
&
If you procrastinate, you keep leaving things you should do until later, often because you do not want to do them.
\\

\hline
can't hack it
&
If you say that someone can't hack it or couldn't hack it, you mean that they do not or did not have the qualities needed to do a task or cope with a situation.
\\

\hline
phobia
&
A phobia is a very strong irrational fear or hatred of something.
\\

\hline
cold call
&
If someone makes a cold call, they phone or visit someone they have never contacted, without making an appointment, in order to try and sell something.
\\

\hline
untempered
&
Not moderated or made more acceptable
\\

\hline
\multicolumn{2}{|l|}{\textbf{Chapter 5 Be Dominant!}}\\

\hline
growl
&
When a dog or other animal growls, it makes a low noise in its throat, usually because it is angry.
\\

\hline
a/the pecking order
&
the order of importance in relation to one another among the members of a group
\\

\hline
red in tooth and claw
&
behaving competitively and ruthlessly
\\

\hline
underling
&
You refer to someone as an underling when they are inferior in rank or status to someone else and take orders from them. You use this word to show that you do not respect someone.
\\

\hline
unbridled
&
If you describe behaviour or feelings as unbridled, you mean that they are not controlled or limited in any way.
\\

\hline
etching
&
An etching is a picture printed from a metal plate that has had a design cut into it with acid.
\\

\hline
defer to
&
If you defer to someone, you accept their opinion or do what they want you to do, even when you do not agree with it yourself, because you respect them or their authority.
\\

\hline
scowl
&
When someone scowls, an angry or hostile expression appears on their face.
\\

\hline
vulture
&
A vulture is a large bird which lives in hot countries and eats the flesh of dead animals. 秃鹫
\\

\hline
squabble
&
When people squabble, they quarrel about something that is not really important.
\\

\hline
prance
&
When a horse prances, it moves with quick, high steps.
\\

\hline
posture
&
You can say that someone is posturing when you disapprove of their behaviour because you think they are trying to give a particular impression in order to deceive people.
\\

\hline
tweed
&
Tweed is a thick woollen cloth, often woven from different coloured threads. 粗花呢
\\

\hline
pinstripe
&
Pinstripes are very narrow vertical stripes found on certain types of clothing. Businessmen's suits often have pinstripes. 白色细条纹
\\

\hline
hog
&
If you hog something, you take all of it in a greedy or impolite way.
\\

\hline
way to go
&
You can say 'Way to go' to show that you are pleased or impressed by something someone has done.
\\

\hline
salve
&
If you do something to salve your conscience, you do it in order to feel less guilty.
\\

\hline
vie
&
If one person or thing is vying with another for something, the people or things are competing for it.
\\

\hline
apocryphal
&
An apocryphal story is one which is probably not true or did not happen, but which may give a true picture of someone or something.
\\

\hline
the crunch
&
the critical moment or situation
\\

\hline
fiscal
&
Fiscal is used to describe something that relates to government money or public money, especially taxes.
\\

\hline
leash
&
A dog's leash is a long thin piece of leather or a chain, which you attach to the dog's collar so that you can keep the dog under control.
\\

\hline
clobber
&
If you clobber someone, you hit them.
\\

\hline
solicitous
&
A person who is solicitous shows anxious concern for someone or something.
\\

\hline
covet
&
If you covet something, you strongly want to have it for yourself.
\\

\hline
deference (see \textit{defer to} above)
&
Deference is a polite and respectful attitude towards someone, especially because they have an important position.
\\

\hline
animosity
&
Animosity is a strong feeling of dislike and anger.
\\

\hline
\multicolumn{2}{|l|}{\textbf{Chapter 6 Defend the Territory}}
\\

\hline
placid
&
A placid person or animal is calm and does not easily become excited, angry, or upset.
\\

\hline
cantaloupe
&
A cantaloupe is a type of melon.
\\

\hline
snort
&
When people or animals snort, they breathe air noisily out through their noses. People sometimes snort in order to express disapproval or amusement.
\\

\hline
usher
&
If you usher someone somewhere, you show them where they should go, often by going with them.
\\

\hline
recliner
&
A recliner is a type of armchair with a back that can be adjusted to slope at different angles.
\\

\hline
put one's finger on sth
&
If you put your finger on something, for example a reason or problem, you see and identify exactly what it is.
\\

\hline
Young Turk
&
a progressive, revolutionary, or rebellious member of an organization, political party, etc, esp one agitating for radical reform
\par
a member of an abortive reform movement in the Ottoman Empire, originally made up of exiles in W Europe who advocated liberal reforms. The movement fell under the domination of young Turkish army officers of a nationalist bent, who wielded great influence in the government between 1908 and 1918
\\

\hline
stultify
&
If something stultifies you, it makes you feel empty or dull in your mind, because it is so boring.
\\

\hline
esprit de corps
&
Esprit de corps is a feeling of loyalty and pride that is shared by the members of a group who consider themselves to be different from other people in some special way.
\\

\hline
fuss
&
If you fuss, you worry or behave in a nervous, anxious way about unimportant matters or rush around doing unnecessary things.
\\

\hline
huffy
&
Someone who is huffy is obviously annoyed or offended about something.
\\

\hline
petty
&
If you describe someone's behaviour as petty, you mean that they care too much about small, unimportant things and perhaps that they are unnecessarily unkind.
\par
You can use petty to describe things such as problems, rules, or arguments which you think are unimportant or relate to unimportant things.
\\

\hline
innate
&
An innate quality or ability is one which a person is born with.
\\

\hline
turf
&
Turf is short, thick, even grass.
\par
A turf is a small rectangular piece of grass which you lay on the ground in order to make a lawn.
\par
Someone's turf is the area which is most familiar to them or where they feel most confident.
\\

\hline
expenditure
&
Expenditure is the spending of money on something, or the money that is spent on something.
\\

\hline
barony
&
A barony is the rank or position of a baron. 男爵爵位
\\

\hline
upshot
&
The upshot of a series of events or discussions is the final result of them, usually a surprising result.
\\


\hline
make do
&
to get along, or manage, with what is available
\\

\hline
morale
&
Morale is the amount of confidence and cheerfulness that a group of people have.
\\

\hline
manger
&
A manger is a low open container which cows, horses, and other animals feed from when indoors.
\\

\hline
malignant
&
A malignant tumour or disease is out of control and likely to cause death.
\par
If you say that someone is malignant, you think they are cruel and like to cause harm.
\\


\hline
fall on deaf ears
&
to be ignored or pass unnoticed
\\

\hline
entrench
&
If something such as power, a custom, or an idea is entrenched, it is firmly established, so that it would be difficult to change it.
\\

\hline
clobber
&
If a person or company is clobbered by something, they are very badly affected by it.
\\

\hline
petty cash
&
Petty cash is money that is kept in the office of a company, for making small payments in cash when necessary.
\\

\hline
incarnate
&
If you say that someone is a quality incarnate, you mean that they represent that quality or are typical of it in an extreme form.
\\

\hline
stranglehold
&
To have a stranglehold on something means to have control over it and prevent it from being free or from developing.
\\

\hline
impeccable
&
If you describe something such as someone's behaviour or appearance as impeccable, you are emphasizing that it is perfect and has no faults.
\\

\hline
bash
&
A bash is a party or celebration, especially a large one held by an official organization or attended by famous people.
\\

\hline
foregoing
&
You can refer to what has just been stated or mentioned as the foregoing.
\\

\hline
come to grips with sth
&
If you come to grips with a problem, you consider it seriously, and start taking action to deal with it.
\\

\hline
retaliate
&
If you retaliate when someone harms or annoys you, you do something which harms or annoys them in return.
\\

\hline
\multicolumn{2}{|l|}{\textbf{Chapter 7 GET THE MATE!}}
\\

\hline
infatuation
&
If you have an infatuation for a person or thing, you have strong feelings of love or passion for them which make you unable to think clearly or sensibly about them.
\\

\hline
banter
&
Banter is teasing or joking talk that is amusing and friendly.
\\

\hline
discreet
&
If you are discreet about something you are doing, you do not tell other people about it, in order to avoid being embarrassed or to gain an advantage.
\\

\hline
adrenaline
&
a hormone that is secreted by the adrenal medulla in response to stress and increases heart rate, pulse rate, and blood pressure, and raises the blood levels of glucose and lipids. It is extracted from animals or synthesized for such medical uses as the treatment of asthma. Chemical name: aminohydroxyphenylpropionic acid; formula: $C_{9}H_{13}NO_{3}$.
\\

\hline
deference
&
Deference is a polite and respectful attitude towards someone, especially because they have an important position.
\\

\hline
confide
&
If you confide in someone, you tell them a secret.
\\

\hline
overt
&
An overt action or attitude is done or shown in an open and obvious way.
\\

\hline
precipitate
&
If something precipitates an event or situation, usually a bad one, it causes it to happen suddenly or sooner than normal.
\\

\hline
elation
&
Elation is a feeling of great happiness and excitement about something that has happened.
\\

\hline
cut out
&
If an engine cuts out, it suddenly stops working.
\\

\hline
surreptitious
&
A surreptitious action is done secretly.
\\

\hline
stake out
&
If you stake out a position that you are stating or a claim that you are making, you are defending the boundaries or limits of the position or claim.
\\

\hline
off-limits
&
If an area or a place is off limits, you are not allowed to go there.
\\

\hline
vacillate
&
If you vacillate between two alternatives or choices, you keep changing your mind.
\\

\hline
apathy
&
You can use apathy to talk about someone's state of mind if you are criticizing them because they do not seem to be interested in or enthusiastic about anything.
\\

\hline
snide
&
A snide comment or remark is one which criticizes someone in an unkind and often indirect way.
\\

\hline
suspicion
&
Suspicion or a suspicion is a belief or feeling that someone has committed a crime or done something wrong.
\\

\hline
tone down
&
If you tone down something that you have written or said, you make it less forceful, severe, or offensive.
\\

\hline

\end{longtable}
\end{center}
\end{document}
